\chapter{Lightweight Real-Time Feature Monitoring} \label{chap:my-work} \minitoc

%what we want (reiterate)
%A reference aggregation snapshot to compare with a streaming aggregation that must be kept in real-time and constant in memory usage

\section*{Outlier detection methods}
%using data mining outlier stuff, too heavy

\section*{Sliding window aggregations}
%- build a aggregation snapshot of the reference and compare with streaming aggregations
 %       - challenges: 
  %          generic SWAG algorithms grow linearly in space regarding window size
   %         defining the set of aggregations that can be done at feature level that represent the current state well enough for comparison and change detection
    %        defining the threshold of deviation between the reference aggregation values and current streaming aggregations to raise alerts
     %   - mention challenge 1 can be solved with Probabilistic Data Structures and EMAs
      %  - there is no trivial solution for the others
       %     - present an example (2 time-series same max min median and std. but different)

\section{Numerical feature monitoring: a two-phased method}

%- introduce method as two-phased method (batch and streaming)

\subsection{Final goal}
%- what is our goal in the streaming phase?
           % - alert when streaming fields/features/variables' distribution changes considerably
          %  - need to choose an aggregation that allows us to know the distribution of variables -> histogram for each feature
               % - lightweight -> approximated EMA-based histogram (recall EMAs background), each bin count is an EMA-count aggregation
               % - describe an EMA approximated histogram and how it discounts old events, the simulated window size, ...
           % - we can compute an exact histogram over the reference period because it is done in batch, where resources like RAM or time are not heavily constrained in constrast to streaming systems
           % - given 2 histograms (ref hist and stream hist) we can apply distance metrics (such as a JSD)
             %   - worth mentioning other metrics can be used (Wasserstein distance, Kolmogorov–Smirnov, ...)
         %   - but how to threshold this "distance" value? ----> we need to know the distribution of "distance" values
             %   - how is this accomplished
             %   - agressive sampling within ref period of smaller periods (equal in size to the stream simulated window)
              %  - give a practical use-case where it makes a diference:
                 %   - left-skewed distro vs right-skewed distro ---> threshold is very different (95 or 99 percentile for example)

\subsection{Batch analysis phase}

%- batch phase: build reference exact histogram and distribution of distance values
          %  1. compute ref histogram (for each feature)
                %- equal heights/counts per bin, different size bins ---> better adjusted to wildly varying distributions
         %   2. make samples of length equal to streaming window size, build histograms for each using bins from reference histogram to make later accurate comparisons between ref hist and sample hist
             %   - why multiple sample histograms?
                   % encode distributions over smaller time periods since there may be pattern changes within the large reference period
             %   - why an approximated histogram for the samples as well despite being done in batch?
                 %   mimic the streaming approximated histogram and the test that will be performed:
                     %   - batch: sample app. histogram vs ref hist
                    %    - stream: streaming app. histogram vs ref hist
         %   3. compute the distance between each sample's histogram and the reference one (for each feature)
          %  4. we now have a list of distance values, and know their expected distribution (for each feature)
              %  - this is important since a test value obtained measuring the distance between the reference period and other window
                %can be compared to this distribution (again probably give the left and right skewed example)

          %  By the end of the batch phase we have:
               % for each feature:
                    %reference histogram (for later comparison with streaming histogram)
                    %list/histogram of distance values
                    %last sample's histogram ---> used as burn-in period for target histogram
                    
\subsection{Streaming phase}

%- online phase:
          %  for each feature:
            %    1. initialize the streaming histogram as the last sample's histogram from the batch phase (burn-in period)
            
         %   for each event:
               % for each field/feature:
                  %  1. update the feature's streaming histogram (recall the update method)
                  %  2. compute the distance between the reference histogram and the target histogram
                   % 3. test the percentile the distance value fits according to the feature's distances list and alert if above a certain percentile (say the 99th)

           % introduce the multiple testing correction problem:
              %  "the more inferences are made, the more likely erroneous inferences are to occur"
             %   we are making multiple hypothesis testing when computing a distance and checking the percentile for each feature

                %introduce the Holm-Bonferroni multiple test correction method
            
            %So in reality we have:

          %  for each event:
               % for each field/feature:
                 %   1. update the feature's streaming histogram (recall the update method)
                  %  2. compute the distance between the reference histogram and the target histogram
                  %  3. apply holm-bonferroni for all feature's distances:
                     %   - each percentile will be a p-value
                    %    - order features by p-value
                    %    - check if p-value > FWER / (m + 1 - k)
                          %  if so then feature is divergent
                          
\subsection{Summary}