\chapter*{Resumo}
Muitos sistemas de monitorização de \textit{streams} de dados em tempo real são estáticos uma vez colocados em produção. Os engenheiros desses sistemas configuram-nos sob o pressuposto de que dados futuros que fluem pelo sistema seguirão aproximadamente a mesma distribuição que os dados vistos anteriormente. Fazem-no conscientemente, com as ferramentas disponíveis, tendo em mente que no futuro poderão precisar de reconfigurar o sistema. Assim, mesmo que a configuração inicial do sistema seja uma das melhores, ao longo do tempo, devido a mudanças nos padrões de dados, o desempenho do sistema em produção decai gradualmente. A detecção de mudança de padrões de dados refere-se ao processo de encontrar padrões em dados que não estão em conformidade com o comportamento esperado ou usual. Uma detecção oportuna e precisa dos desvios dos padrões de dados permite que medidas imediatas sejam tomadas. Assim, o problema em questão é determinar quando reconfigurar o sistema, \textit{e.g.}, um modelo de \textit{Machine Learning}, com base numa análise dos desvios de padrões da \textit{stream} de dados.

%Approach/Solution and %Main contributions
Nesta tese, projetamos um método que utiliza agregações de \textit{streaming} leves e funções de divergência para alertar sobre desvios nos padrões de dados em tempo real, na presença de grandes volumes de dados que fluem a alta velocidade, são altamente \textit{skewed} e sazonais. Desenvolvemos um método de duas fases e duas janelas: primeiro, realizamos uma análise \textit{batch} numa janela de referência e, em seguida, uma análise \textit{streaming} sobre uma janela deslizante. Agregamos o conteúdo de ambas as janelas usando um histograma aproximado baseado em Médias Móveis Exponenciais (MMEs). Fazemos isto para cada evento e cada um dos seus campos, também denominados \textit{features} no nosso contexto. Medimos a divergência entre os histogramas de referência e os das janelas deslizantes baseados em MMEs para cada \textit{feature} com a função de divergência de \textit{Jensen-Shannon}, mas outras funções podem ser usadas. Para cada \textit{feature}, descobrimos o valor de probabilidade para cada medida de divergência e aplicamos a correção de múltiplo teste de Holm-Bonferroni para descobrir quais os valores de probabilidade mais baixos e as \textit{features} associados. Geramos alertas  para cada \textit{feature} com base num limite de probabilidade definido pelo administrador.

Avaliamos o nosso método através de uma série de testes, alguns com conjuntos de dados sintéticos e outros com dados reais. Dividimos ainda os testes em testes de análise a uma única \textit{feature} e tests de análise com múltiplas \textit{features}. O nosso método detectou com precisão as anomalias introduzidas nos testes com conjuntos de dados sintéticos, mantendo um alto \textit{throughput}, tanto para análises com uma única ou múltiplas \textit{features}. No entanto, os testes com dados reais não foram tão precisos. Apesar de não sabermos as razões específicas para isso, passamos os próximos testes para trabalhos futuros e formulamos um conjunto de hipóteses que pretendem explicar o sucedido e, portanto, são dignas de serem exploradas em seguida. O nosso conjunto de testes apoia a afirmação de que métodos de agregação de janelas deslizantes e métodos de divergência de distribuição podem ser combinados para detectar alterações nos padrões de dados em cenários de \textit{streaming} com complexidade temporal e de memória constante.
