\chapter*{Resumo}

A deteção de mudanças de padrões nos dados tem sido investigada nas mais diversas áreas e aplicações e diz respeito ao processo de encontrar padrões nos dados que não correspondam ao esperado ou usual. A deteção atempada e correta destas mudanças de padrões permite que medidas imediatas sejam tomadas, prevenindo perdas de rendimento e mantendo a reputação da empresa entre os seus clientes. Como tal e tendo isto em conta, muitas empresas constroem sistemas de deteção de anomalias. Desta maneira, administradores do sistema são capazes de proactivamente ajustar e prevenir potenciais falhas nos seus serviços.

Por exemplo, o sistema de deteção de anomalias da Yahoo serve o propósito de monitorizar e alertar \textit{streams} de eventos temporais em múltiplos casos de uso da Yahoo. Na Microsoft, o seu sistema interno de deteção de anomalias monitoriza milhões de métricas provenientes de diferentes serviçoes internos, permitindo que os engenheiros os resolvam em tempo real antes que estes problemas adquiram proporções maiores.

Muitos sistemas de monitorização em tempo real são estáticos quando implementados num ambiente de produção. Os engenheiros responsáveis por tais sistemas configuram-nos assumindo que os dados que futuramente fluirão pelo sistema se assemelham aos dados observados até ao momento. Eles fazem-no conscientemente e o melhor que sabem com as ferramentas de que dispoem, mantendo sempre em mente que no futuro possivelmente terão de reconfigurar o sistema. Assim, apesar da configuração inicial do sistema ser a que melhor se adequa, ao longo do tempo, devido a desvios nos padrões de dados, o desempenho do sistema estático incialmente implementado decai gradualmente.


Desta forma, pretendemos avaliar o uso de métodos de análise de \textit{streams} eficientes em termos de memória --- \textit{i.e} métodos que usam pouca memóra e calculam os resultados esperados em tempo real --- com o intuito de detetar desvios nos padrões de dados na presença de grandes volumes de dados com padrões sazonais que fluem pelo sistema a grande velocidade.

\vspace*{10mm}\noindent
\textbf{Keywords}: tempo real, streaming, leves, monitorização, volume, velocidade, variedade, sazonalidade