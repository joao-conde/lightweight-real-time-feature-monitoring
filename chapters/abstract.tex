\chapter*{Abstract}
 
Data pattern shift detection has been researched within diverse areas and application domains and refers to the process of finding patterns in data that do not conform to expected or usual behavior. Accurate and timely detection of data pattern deviations allows for immediate measures to be taken, preventing loss of income and maintaining the companies' reputation among customers.
As such, many large companies take this into consideration and deploy data anomaly components in their systems. In this way, system administrators can proactively adjust and prevent potential service failures.

For example, Yahoo's anomaly detection system fulfills its purpose of monitoring and raising alerts on time-series events for several of Yahoo's use cases. At Microsoft, their custom, internally made anomaly detection system monitors millions of metrics coming from a variety of different internal services, enabling engineers to solve live on-site issues before they scale and become bigger ones.

Many real-time stream monitoring systems are static once deployed in a production environment. Engineers of said systems usually configure them under the assumption that future data flowing through the system follows roughly the same distribution as previously seen data. They do so consciously, to the best of their knowledge and available tools, keeping in mind that in the future they may need to reconfigure the system. Thus, even though the initial configuration of the system may be one of the best fits, over time, due to data pattern shifts, the initially deployed static system's performance gradually deteriorates.

With this thesis, we set out to design and evaluate a system that makes use of lightweight streaming analytical methods to detect deviations in data patterns while in the presence of large volumes of high velocity, highly skewed, seasonal data. 

\vspace*{10mm}\noindent
\textbf{Keywords}: real-time, streaming, lightweight, monitoring, volume, velocity, variety, seasonal

