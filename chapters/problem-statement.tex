\chapter{Problem Statement} \label{chap:statement} \minitoc

With this Chapter, we further discuss the problem we aim to solve with this Thesis making use of concepts discussed in Chapter \ref{chap:background} and explaining why none of the reviewed methods from Chapter \ref{chap:sota} works for our use case.

\section{State of the art issues}
Aggregations, specifically sliding window ones, were introduced in Chapter \ref{chap:background} along with several windowing techniques. In Chapter \ref{chap:sota} we presented state of the art sliding window aggregation algorithms as well as outlier detection methods. However, the reviewed methods were deemed unfit for our use-case because none of them simultaneously showed all of the below characteristics:

\begin{itemize}
    \item \textbf{Low memory consumption:} need for a low memory footprint system that avoids linear growth relative to the sliding window size used
    
    \item \textbf{Low time complexity:} need for a time efficient method that updates the sliding window aggregation and raises alerts in constant time
    
    \item \textbf{Fixed reference period:} the reference window must be kept the same throughout run-time, the opposite of an online learning algorithm
    
    \item \textbf{Sliding window maintainability:} possibility to evict old events and insert new ones while updating the aggregation state
    
    \item \textbf{Explainable alerts:} understand why each alert was raised
\end{itemize}


\section{Research Questions}
The goal of this Thesis is to develop a lightweight stream monitoring system that detects changes in features distribution between a reference period and a sliding window in real-time.

To that end, we devise a series of research questions to be answered by the end of this Thesis:

\begin{enumerate}
    \item \textbf{RQ1:}
    \item \textbf{RQ2:}
    \item \textbf{RQ3:}
\end{enumerate}