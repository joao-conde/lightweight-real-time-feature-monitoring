%%%% Proceedings format for most of ACM conferences (with the exceptions listed below) and all ICPS volumes.
\documentclass[sigconf]{acmart}
\usepackage{graphicx}
% Used for displaying a sample figure. If possible, figure files should
% be included in EPS format.
%
% If you use the hyperref package, please uncomment the following line
% to display URLs in blue roman font according to Springer's eBook style:
% \renewcommand\UrlFont{\color{blue}\rmfamily}
\usepackage{algorithm}
\usepackage[noend]{algpseudocode}
\usepackage{amsmath}
\usepackage{caption}
\usepackage{subcaption}
\usepackage[group-separator={,}]{siunitx}

%%%% As of March 2017, [siggraph] is no longer used. Please use sigconf (above) for SIGGRAPH conferences.

%%%% Proceedings format for SIGPLAN conferences 
% \documentclass[sigplan, anonymous, review]{acmart}

%%%% Proceedings format for SIGCHI conferences
% \documentclass[sigchi, review]{acmart}

%%%% To use the SIGCHI extended abstract template, please visit
% https://www.overleaf.com/read/zzzfqvkmrfzn

%
% defining the \BibTeX command - from Oren Patashnik's original BibTeX documentation.
\def\BibTeX{{\rm B\kern-.05em{\sc i\kern-.025em b}\kern-.08emT\kern-.1667em\lower.7ex\hbox{E}\kern-.125emX}}
    
% Rights management information. 
% This information is sent to you when you complete the rights form.
% These commands have SAMPLE values in them; it is your responsibility as an author to replace
% the commands and values with those provided to you when you complete the rights form.
%
% These commands are for a PROCEEDINGS abstract or paper.
\copyrightyear{}
\acmYear{}
\setcopyright{none}
\acmConference[KDD-ADF-2019]{2nd KDD Workshop on Anomaly Detection in Finance}{August 5, 2019}{Anchorage, Alaska, USA}
\acmBooktitle{}
\acmPrice{}
\acmDOI{}
\acmISBN{}
\acmISBN{}

%
% These commands are for a JOURNAL article.
%\setcopyright{none}
%\acmJournal{TOG}
%\acmYear{2018}\acmVolume{37}\acmNumber{4}\acmArticle{111}\acmMonth{8}
%\acmDOI{10.1145/1122445.1122456}

%
% Submission ID. 
% Use this when submitting an article to a sponsored event. You'll receive a unique submission ID from the organizers
% of the event, and this ID should be used as the parameter to this command.
%\acmSubmissionID{123-A56-BU3}

%
% The majority of ACM publications use numbered citations and references. If you are preparing content for an event
% sponsored by ACM SIGGRAPH, you must use the "author year" style of citations and references. Uncommenting
% the next command will enable that style.
%\citestyle{acmauthoryear}

%
% end of the preamble, start of the body of the document source.
\begin{document}

%
% The "title" command has an optional parameter, allowing the author to define a "short title" to be used in page headers.
\title{Lightweight Real-Time Feature Monitoring}

%
% The "author" command and its associated commands are used to define the authors and their affiliations.
% Of note is the shared affiliation of the first two authors, and the "authornote" and "authornotemark" commands
% used to denote shared contribution to the research.

\author[Conde]{João Conde}
\email{joao.conde@feedzai.com}
\affiliation{%
  \institution{Feedzai}
}

\author[Sampaio]{Marco O. P. Sampaio}
\email{marco.sampaio@feedzai.com}
\affiliation{%
  \institution{Feedzai}
}

\author[Cardoso]{Pedro Cardoso}
\email{pedro.cardoso@feedzai.com}
\affiliation{%
	\institution{Feedzai}
}

\author[Ribeiro]{Pedro Manuel Pinto Ribeiro}
\email{pribeiro@dcc.fc.up.pt}
\affiliation{%
  \institution{FCUP}
}

\author[Restivo]{André Monteiro de Oliveira Restivo}
\email{arestivo@fe.up.pt}
\affiliation{%
	\institution{FEUP}
}

%
% By default, the full list of authors will be used in the page headers. Often, this list is too long, and will overlap
% other information printed in the page headers. This command allows the author to define a more concise list
% of authors' names for this purpose.

%
% The abstract is a short summary of the work to be presented in the article.
\begin{abstract}
% The problem
Many real-time stream monitoring systems are static once deployed in a production environment. The engineers of those systems configure them under the assumption that future data flowing through the system roughly follows the same distribution as previously seen data. They do so consciously, to the best of their knowledge and available tools, keeping in mind that in the future they may need to reconfigure the system. Thus, even though the initial configuration of the system may be one of the best fits, over time, due to data pattern shifts, the initially deployed static system's performance gradually deteriorates. Data pattern shift detection refers to the process of finding patterns in data that do not conform to expected or usual behavior. Accurate and timely detection of data pattern deviations allows for immediate measures to be taken. Thus, the problem at hand is to determine when to reconfigure the system, \textit{e.g.}, a Machine Learning model, based on an analysis of the drifts in the stream of data.

%Approach/Solution and %Main contributions
In this thesis, we design a method that makes use of lightweight streaming aggregations and distribution divergence functions to alert about deviations in data patterns in real-time, while in the presence of large volumes of high velocity, highly skewed and seasonal data. We develop a two-phased and two-windowed method: first, we perform a batch analysis on a reference window and then a stream analysis over a sliding streaming window. We aggregate the contents of both windows using an approximate histogram aggregation based on Exponential Moving Averages (EMAs). We do this for each event and each of its fields, also denominated features in our context. We measure the divergence between both reference and sliding window EMA-based histograms for each feature with the Jensen–Shannon Divergence function, but other distance functions could be used. For each feature, we find out the probability value of each divergence measure and apply the Holm-Bonferroni multiple test correction to find out which probability values are lowest and their associated features. We generate alerts based on a user-defined probability threshold, for each feature.

We evaluate our method through a series of tests, some with synthetic datasets and some with real data. We further split the tests into single and multi-feature analysis. Our method accurately detected the introduced anomalies in the experiments using synthetic datasets while maintaining high throughput, for single and multi-feature analysis. However, experiments with real data were not as accurate. Despite not knowing the specific reasons for that, we defer further investigation to future work and formulate a set of hypotheses that might explain it and hence are worthy of pursuing next. Our set of experiments supports the claim that sliding window aggregations and distribution divergence methods can be combined to detect data pattern shifts in streaming scenarios with constant time and memory complexity.
\end{abstract}

%
% The code below is generated by the tool at http://dl.acm.org/ccs.cfm.
% Please copy and paste the code instead of the example below.
%
\begin{CCSXML}
<ccs2012>
 <concept>
  <concept_id>10010520.10010553.10010562</concept_id>
  <concept_desc>Computer systems organization~Embedded systems</concept_desc>
  <concept_significance>500</concept_significance>
 </concept>
 <concept>
  <concept_id>10010520.10010575.10010755</concept_id>
  <concept_desc>Computer systems organization~Redundancy</concept_desc>
  <concept_significance>300</concept_significance>
 </concept>
 <concept>
  <concept_id>10010520.10010553.10010554</concept_id>
  <concept_desc>Computer systems organization~Robotics</concept_desc>
  <concept_significance>100</concept_significance>
 </concept>
 <concept>
  <concept_id>10003033.10003083.10003095</concept_id>
  <concept_desc>Networks~Network reliability</concept_desc>
  <concept_significance>100</concept_significance>
 </concept>
</ccs2012>
\end{CCSXML}

%\ccsdesc[500]{Computer systems organization~Embedded systems}
%\ccsdesc[300]{Computer systems organization~Redundancy}
%\ccsdesc{Computer systems organization~Robotics}
%\ccsdesc[100]{Networks~Network reliability}

%
% Keywords. The author(s) should pick words that accurately describe the work being
% presented. Separate the keywords with commas.
\keywords{data streams, monitoring, real-time, lightweight, concept drift}

%
% This command processes the author and affiliation and title information and builds
% the first part of the formatted document.
\maketitle

\section{Introduction}
Nice Intro

\section{Related Work}
\label{sec:RelatedWork}
Nice related work

\section{Method}
\label{sec:Method}

Such method wow

\section{Experiments}
\label{sec:Experiments}

Much scientist

\section{Discussion}
\label{sec:Discussion}

Fight

\section{Conclusions}
\label{sec:Conclusions}
\cite{ApacheFlink}
Thesis done, PogU

%
% The next two lines define the bibliography style to be used, and the bibliography file.
\bibliographystyle{ACM-Reference-Format}
\bibliography{myrefs.bib}

\end{document}
